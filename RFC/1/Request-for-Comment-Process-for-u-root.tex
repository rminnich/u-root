\documentclass{article}
\usepackage[utf8]{inputenc}

\title{Request for Comment Process for u-root}
\author{Ron Minnich}
\date{October 2019}

\usepackage{natbib}
\usepackage{graphicx}

\begin{document}

\maketitle

\section{Introduction}
We need a Request For Comment (RFC) process for u-root. This document outlines it.

\section{Motivation}
u-root has grown far faster, with wider usage, than we ever anticipated.
It is straining to accommodate the growth.
For example, the init system is not really adequate, but we want to avoid the mistakes of the past. 
We need a way to present new ideas for u-root, get comments, come to agreement, and document the result. 

It makes sense to retain the RFCs in u-root itself, under git control, and use the github PR process to modify and approve them.

We tried something like this years ago in coreboot, but lacking the process proposed in this RFC, it never worked very well.
I'm writing this document in hopes that building some rules will make it succeed.

\section{Process}
This RFC process is loosely based on the IETF RFC process\cite{RFC0001}.
A user will submit a PR for an RFC, sequentially numbered so referring to it is easy,
and we will use the github comment mechanism to refine it until it meets approval. 

Just what constitutes approval is tricky, but we might consider requiring, say, three organizations to sign off; or we might have the Technical Steering Committee be the gatekeeper.

\section{Format}
RFCs should follow the Title, Introduction, Motivation pattern for the front matter. 
Authors names can be listed at the front. 
Authors names, and affiliations, can be listed at the end. 
RFCs shall be written in Latex using the article documentclass.
If you are not sure of how to use Latex, consider using overleaf.com (a sort of Google docs for Latex), or for local use, Texmaker.
For citations, Google Scholar is very helpful for getting bibtex format entries.
Each RFC will be contained in a directory with the directory name in the format
RFC number-title-with-hyphens-for-spaces, e.g. this document is
RFC/1/Request-for-Comment-Process-for-u-root.

Common bibliograpy entries will be maintained at in the RFC directory itself as RFC/rfc.bib and can be referenced with cite commands; see the source of this document for an example.

\section{Helpful hints}
In general, in your latex files, use one sentence per line to make git reviews easier.

\section{authors}
Ron Minnich\\Google

\bibliographystyle{plain}
\bibliography{../rfc}
\end{document}
